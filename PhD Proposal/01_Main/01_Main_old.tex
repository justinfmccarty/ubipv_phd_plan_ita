
\section{Introduction}\

The clear and present, as well as uncertain and imminent, threats posed by the destabilisation of the climate as our species has known it require a significant effort in adaptation of societal mechanics and norms, as well as mitigation of existing sources of greenhouse gasses (GHGs) and abatement of potential future sources of GHGs [1.5C report]. The majority of adaptation, mitigation, and abatement efforts exist in the context of cities due to the majority share demand placed on energy supplies and the resulting majority share of global GHG emissions [WUP 2018 Report]. A majority share of the world's population is also contained within cities [pending]. 


\subsection{Motivation}\

As new modes of power production are introduced, there are benefits to co-locating them in demand centres in terms of consumption but also in mitigating heat gain [multi-energy systems paper]. As well the current practice of greenfield development for large-scale solar renewable energy production can be mitigated by relying on surfaces found within cities [source on comparing existing pv production in urban regions and hinterland]. Additionally the cost burden of these technologies can be shared if integrated into urban surfaces. 


\subsubsection{location \& surface cobenefits}\

\vspace{2cm}


\subsubsection{land use considerations}\

\vspace{2cm}


\subsubsection{cost of pv}\

\vspace{2cm}


\subsection{Solar Technology on Urban Surfaces}\
\vspace{2cm}

Active Solar Energy Technologies (ASETs) span a collection solar power conversion technologies from simplistic monocrystalline panels to more complex solar cooling technologies. The tendency away from a buildings specific nomenclature is that the urban realm offers many more surfaces an opportunities for integration of solar energy than merely the surfaces of buildings. There are rail yards which host significant structural potential, plazas and urban farms which require shading, and much more. To forget about these spaces would be to reduce the potential of our cities to be utilised in plans for carbon-free electricity generation. The expansion from just photovoltaics is in line with this thinking. Photovoltaics have the potential to solve issue of electricity generation, but may not always be the correct solution. Inn some instances an urban space may need hot air, for which a solar air heater may be more appropriate or an indoor space may require cooling, for which a fluid based system is required in desiccant and ejector cycle systems. It is expected though that BIPV will make up the majority 

\subsection{Problem Background \& Statement}\

Urban ASET research is a nascent field. The availability of these technologies is only now approaching grid parity in may parts of the world making the economic case of integration into urban surfaces economically feasible. Urban environments 

Deployment of ASETs has yet to reach the scale of installation required for meaningful decarbonisation. As well, many of the existing technologies are not widely known to potential end users. Simultaneously it is not known if present and future energy demand in the urban environment can be met with existing ASETs. 

While the energy from the sun is effectively unlimited, solar panels or rather their components are constrained resources. Due to the rare-earth minerals that are required to construct them and the imposition of carbon budget efficient deployment is necessary if they are to be highly rewarding low-toxicity and emissions mitigation solutions [check the FCL documentation on this, may need to write this paper]. While this does not mean restricting the use of solar panels wholesale, it implies understanding that some regions of the world would see more significant mitigation from and array than another region of the world would for the same array. For instance, if you could only install a 10kW array, would it be more effective (from an emissions standpoint) to install it in Singapore where the carbon emissions factor for the grid is 410gCO2e/kWh delivered or in Switzerland where the carbon emissions factor for the grid is 14gCO2e/kWh delivered. This seems a simple question, but what is the addition of the array offsetting? Is it peak  electricity consumption? Is it natural gas in boilers for heating (uncommon in Singapore)? Is it servicing the addition of new electric vehicles? The questions for how this simple array is integrated into the urban energy flow is complex and requires a significant investigation.

\subsubsection{The deployment gap}\

Technologies such as BIPV have been widely accepted in decarbonisation plans and speculation on mitigation efforts as widely applicable. Upon examination of deployment patters 

\vspace{2cm}


\subsubsection{Capacity \& robustness in meeting demand}\

\vspace{2cm}


\subsubsection{The Carbon Budget \& ASETs}\

\vspace{2cm}


\subsection{Research Objectives}\
%(What do you want to achieve, more in general terms? Bullet point list)\\

Based on the problem statement, the objectives, in reverse order are to

\begin{itemize}
	\item Disseminate localized deployment schedules of ASETs for meeting decarbonisation goals. that follow several socioeconomic and energy development scenarios.
	\item Identify and formulate models and methods for urban ASETs performance simulation that are robust at the urban scale. 
	\item Develop an understanding of where existing ASETs are failing either in performance or their ability to be deployed and propose solutions in the form of digital or analog models.
\end{itemize}

\subsection{Hypothesis}\

These objectives...

Urban areas and in particular district envelopes represent an area of great potential for the deployment of solar renewable energy. This is due to the synthesis of existing or planned surfaces which receive reasonable amounts of solar radiation and the level of electricity demand, existing and expected, that occurs beneath these surfaces. Utilising these surfaces for the deployment of ASETs can lead to decarbonisation in line with regional and municipal goals, increased grid resiliency, and allow for adaptation to climate change and to new modes of transport in the city.

By understanding where existing ASETs are succeeding in being deployed and meeting demand, as well as where they are failing in these regards, we can determine what innovations are necessary to ensure urban surfaces are utilized to their maximum potential in the interests laid out above. Thus a hypothesis is formed; different varieties of urban ASETs are more suited to different urban environments. The deployment of these ASETs in the near term (10 to 20 years) in the form of schedules can be formulated under a variety of scenarios accounting for the unknowns of technology acceleration, climate change, and socioeconomic pathways. These schedules form notional recommendations at their largest scale (planetary) and practical recommendations at their most precise scale (the urban envelope), allowing stakeholders to take advantage of the research from a variety of decision making positions.

\subsection{Research Questions}\
%(What questions are you planning to answer?)\\
Research in the interest of testing these hypotheses will be guided by the following research questions:

\begin{itemize}
	\item What ASETs are currently being researched and deployed in the urban environment?
	\item What is missing from existing ASETs?
	\item How should ASETs be modeled at the urban scale?
	\item Where should ASETs be deployed and what is their timeline, given some are novel, to most efficiently mitigate carbon emissions globally, taking into consideration local factors?
\end{itemize}

\subsection{Audience and Delivery of Knowledge}\

\begin{itemize}
    \item Global CC Mitigation \& Adaptation environment (High quality research papers and graphics)
    \item Fellow Urban Scientists (domain specific knowledge through papers)
    \item Stadt Zurich (Summary of relevant findings)
    \item Singapore BE (Summary of relevant findings
    \item Architects (Simple Assessment Web Tool)
\end{itemize}

\subsection{Research Tasks}\

This research will be undertaken across three work packages. The first concerns an effort to characterize existing Urban ASETs, as well as identify and assess the literature relevant to the field. With this I will locate gaps within existing systems which are necessary to meeting urban demand. These novel ASETs will finalize the first work package. The second package, to be undertaken simultaneously to the first concerns the practice of simulating and modeling various ASETs at the urban scale. While urban building energy modeling and urban radiation studies are not a research space in infancy there are still open questions t the appropriate levels of detail necessary to authentically model these technologies across urban districts. The final work package represents a culmination of the research. Existing and novel systems will be used in urban scale simulations and component level validation to assess deployment of ASETs in the near term within the case study districts selected for Zurich and Singapore. From these studies I intend to scale up analysis to a global scale developing near term deployment schedules for the Globe following various energy growth scenarios.  

- There is a differentiation needed between optimally design generation components and systems and optimally designed urban plans and buildings. (What is typically good for passive is typically good for active?)

- 

\subsubsection{WP1: Urban ASETs, existing \& novel}\

- impacts of UHI from components and systems
\vspace{2cm}


\subsubsection{WP2: Simulation \& modeling practices}\

\vspace{2cm}

\subsubsection{WP3: Localised deployment schedules}\

\vspace{2cm}



\subsection{Research Novelty}\

\vspace{2cm}

\section{Methodology}\
%(How do you think this could be achieved?)

% Methodologically, we will research and implement data-driven approaches combined with engineering modelling and simulation to facilitate a framework that supports evidence-based decision making for design, planning and retrofitting of cities. This includes, in an iterative process, environmental life-cycle and cost-benefit considerations as well as aspects of acceptability and design. Methods and findings will be translated to decision making via advanced digital toolsets, principal knowledge on potentials and parameters and resulting heuristics for large scale deployment of BIPV for future low carbon cities.

- in the end, mixed supply models for optimising deployment of PV (including passive options)

\subsection{supervised projects}\

\vspace{2cm}


\subsection{Systematic Literature Review}\

\vspace{2cm}

\subsection{Experiments \& Case Studies}\
\vspace{1cm}
\subsubsection{Altstetten}

\vspace{1cm}

\subsubsection{SG}

\vspace{1cm}

\subsubsection{Catalog}

\vspace{1cm}

\subsubsection{Solar Lab Models}

\vspace{1cm}


\section{Time frame and current progress}\
\noindent\resizebox{\textwidth}{!}{
\begin{ganttchart}[
    x unit=1.0cm,
    y unit title=0.5cm,
    y unit chart=1.25cm,
    vgrid,
    newline shortcut=true,
    bar label node/.append style={align=right}
    ]{1}{16}
	\gantttitle{2020}{2}
	\gantttitle{2021}{4}
	\gantttitle{2022}{4}
	\gantttitle{2023}{4}
	\gantttitle{2024}{2} \\
	\ganttgroup{Step 1: Literature Review}{1}{4} \\
	\ganttgroup[group label node/.append style={align=right}]{Step 2: Numerical \ganttalignnewline Simulation of Envelopes}{1}{10} \\
	\ganttbar{Initial Simulation and Paper}{1}{2} \\
	\ganttbar{Simulation with Python-E+}{6}{8} \\
	\ganttgroup{Step 3: Simulation with Building System}{6}{12} \\
	\ganttgroup{Step 4: Experimental Investigation}{2}{13} \\
	\ganttbar{HoNR ASF Construction}{2}{5} \\
	\ganttbar{Build Sensors and Program}{5}{8} \\
	\ganttbar{Run Measurements}{7}{11} \\
	\ganttbar{Comfort and Acceptance Studies}{8}{14}\\
	\ganttgroup{Step 5: LCA Calculation}{3}{6} \\
	\ganttmilestone{Journal Papers}{5}
	\ganttmilestone{}{10}
	\ganttmilestone{}{8}
	\ganttmilestone{}{13} \\
	\ganttbar{Thesis compiling}{12}{16} \\
	\ganttmilestone{RQE}{7} \\
% 	\ganttlink{elem0}{elem3}
	
\end{ganttchart}
}
\newpage

\section{Expected Results}\
\vspace{1cm}
The expected results from each work package are as w:
\begin{itemize}
	\item Step 1: \vspace{1cm}
	\item Step 2: \vspace{1cm}

\end{itemize}

\vspace{1cm}
\section{Impact}\


\section{Preliminary Work and Publications}
\vspace{1cm}
The following activities have already been completed:

\begin{itemize}
	\item \vspace{1cm}
	\item \vspace{1cm}

\end{itemize}


\section{Signatures}
